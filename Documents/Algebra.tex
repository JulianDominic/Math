\documentclass[../setup.tex]{subfiles}
\graphicspath{{../images}}

\begin{document}

\title{Algebra; A Brief Overview}
\author{Julian Dominic}
\date{\today}
\pagenumbering{gobble}
\maketitle
\clearpage

\pagenumbering{arabic}
\setcounter{page}{1}
\section{Groups}
The integers under the operation of addition, $(\mathbb{Z}, +)$, are a key example of what is known as a \textit{group}. A group consists of an underlying set, which in the case of the integers is $\mathbb{Z}$, coupled with an operation, addition $(+)$ in this case, which allows two members of the set $a$ and $b$ to interact and produce another member, $a + b$, of the same set.  \\ \\
A group operation must be a \textit{binary operation}, like addition, meaning that it involves two elements of the set. What is more, for a binary operation to be a group operation. \\ \\ 
We insist further that the operation satisfies three particular conditions, all of which hold for integer addition: the operation, $+$, must be \textit{associative}, i.e. $a + (b + c) = (a + b) + c$ for any three members $a, b, c$ of the set. \\ \\
There must be an \textit{identity element}, denoted by $0$, which has the property that $a + 0 = 0 + a = a$ always holds. \\ \\
Finally, each member $a$ of the set must have an \textit{inverse} element, denoted here by $-a$, that reverses the effects of adding $a$ in the sense that $a + (-a) = (-a) + a = 0$, the identity element. \\ 

\begin{remark}[Abelian Group]
Integer addition also satisfies the \textit{commutative} law in that $a + b = b + a$. Commutativity is not part of the general definition of a group. However, when the operation of a group $G$ does respect the commutative law, we say that $G$ is an \textit{abelian group}.
\end{remark}

\section{Rings}
\section{Fields}


\end{document}