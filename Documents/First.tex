\documentclass{article}
\usepackage[utf8]{inputenc}
\usepackage{amsmath, amsthm, amsfonts}
\usepackage{graphicx}

\newtheorem{theorem}{Theorem}

\title{First Document}
\author{Julian Liaw}

\begin{document}
\maketitle

\section*{Definitions of $e$}
Let's begin with a formula $e^{i\pi} + 1 = 0$. 

\begin{enumerate}
\item As a \textbf{limit}: 

\begin{equation} 
\begin{split}
\label{limit}
e &= \lim_{n\to\infty} \left(1+ \frac{1}{n}\right)^n \\ &= \lim_{n\to\infty}\frac{n}{\sqrt[n]{n!}} \\ &= \lim_{t \to 0}(1+t)^{\frac{1}{t}} \\
\end{split}
\end{equation}

\item As a \textit{sum}:

\[ e = \sum_{n=0}^{\infty} \frac{1}{n!} \]

\begin{multline}
e^x \approx 1 + x + \frac{x^2}{2!} + \frac{x^3}{3!} + \frac{x^4}{4!} + \frac{x^5}{5!} + \frac{x^6}{6!} + \frac{x^7}{7!} \\ + \frac{x^8}{8!} + \frac{x^9}{9!} + \frac{x^{10}}{10!} + \frac{x^{11}}{11!} + \frac{x^{12}}{12!} + \frac{x^{13}}{13!} + \dots
\end{multline}

\item As a \underline{continued fraction}:

\[ e = 2 + \frac{1}{1+\frac{1}{2+\frac{2}{3+\frac{3}{4+\frac{4}{5+\ddots}}}}} \]
\end{enumerate}

Equation \ref{limit} was really cool

\newpage
\section*{More Tricks}

\begin{table}
\caption{A nifty table!}
\label{niftytable}
\begin{center}
\begin{tabular}{| c | r |}
	\hline
	1 & 2 \\ \hline
	3 & 40000000000000000000000000000 \\ \hline
\end{tabular}
\end{center}
\end{table}

I like table \ref{niftytable}

\begin{figure}
	\centering
	\includegraphics[width=\textwidth]{length}
	\caption{I did it!}
	\label{fig:length}
\end{figure}

I am so proud of figure \ref{fig:length}

\begin{example} You should like and subscribe!
\end{example}

\end{document}