\documentclass[../setup.tex]{subfiles}
\graphicspath{{../images}}

\begin{document}

\title{Notes from How to Prove It}
\author{Julian Dominic}
\date{28 July 2022}
\pagenumbering{gobble}
\maketitle
\clearpage

% ===== Define a preface environment =====
\newcommand{\prefacename}{Preface}
\newenvironment{preface}{
    {\noindent \bfseries \Huge \prefacename}
    \begin{center}
        % \phantomsection \addcontentsline{toc}{chapter}{\prefacename} % enable this if you want to put the preface in the table of contents
        \thispagestyle{plain}
    \end{center}%
}


\preface
I wrote these notes principally FOR understanding, they are meant for future reference for a refresher on what I have learnt. \\
Honestly, I don't have much to say, I just thought that a preface would be cool. Let's start our journey now.



\tableofcontents
\pagenumbering{gobble}
\clearpage

\pagenumbering{arabic}
\setcounter{page}{1}



\section{Sentential Logic}
\subsection{Deductive reasoning and logical connectives}
In an \textit{argument}, we arrive at \textbf{valid} \textit{conclusions} assuming that the \textit{premises} are \textbf{true}. \\ 
Premises and conclusiosn are often referred to as conditions and outcomes respectively. \\
If all the premises are \textbf{true}, then the conclusion should be \textbf{true}. However, for the case where the conclusion is \textbf{false} while the premises are \textbf{true}, the argument is \textbf{invalid}.

\subsubsection{Logical Operators}
{
\centering
\begin{tabular}{| c | c | c |}
\hline
Symbol & Meaning & Description \\
\hline
& & \\
$\lor$ & OR & Disjunction \\
& & \\
$\land$ & AND & Conjunction \\
& & \\
$\lnot$ & NOT & Negation \\
\hline
\end{tabular} \\
}



\subsection{Truth tables}
A truth table must be able to represent all possible combinations of the variables, premises and conclusions.\\
\\
{
\centering
\begin{tabular}{|c|c|c|}
\hline
$P$&$Q$&$P\land Q$ \\
\hline
F&F&F \\
F&T&F \\
T&F&F \\
T&T&T \\
\hline
\end{tabular} \\
} 

\paragraph{} In this case, we see that our variables (or statements), $P$ and $Q$ have their individual column to assign a value -- \textbf{True} or \textbf{False} -- to them. \\
We use our logical operators to make a new statement from $P$ and $Q$ which is $P\land Q$, and assign a value to the new statement. \\
\\
It is important note that the number of variables will dictate the number of rows that the truth table will have. Construct a truth table for the following set of variables, $\{P\}$, $\{P, Q\}$, $\{P, Q, R\}$. \\
The pattern that we find is that as the number of variables increases, the number of rows increases two-fold. \\
\[ \text{Number of Rows} = 2^{\text{Number of Variables}} \]
\\
There are some special truth tables where the column for the conclusion always has the same value (either all \textit{true} or all \textit{false}) for every combination of the variables' values. \\
\\
When the conclusion is always \textit{true}, we say that the conclusion's statement is a \textbf{tautologies}. Construct a truth table for $P\lor\lnot P$.\\
Similarly, when the conclusion is always \textit{false}, we say that the conclusion's statement is a \textbf{contradiction}. Construct a truth table for $P \land\lnot P$. \\

\begin{remark}
\textit{Tautologies} and \textit{Contradictions} are not the only laws that govern logic. Do see the logic document for more.
\end{remark}



\subsection{Variables and sets}




\subsection{Operations on sets}
\subsection{The conditional and biconditional connectives}

\end{document}