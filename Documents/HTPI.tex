\documentclass[../setup.tex]{subfiles}
\graphicspath{{../images}}

\begin{document}

\title{Notes from How to Prove It}
\author{Julian Dominic}
\date{28 July 2022}
\pagenumbering{gobble}
\maketitle
\clearpage

% ===== Define a preface environment =====
\newcommand{\prefacename}{Preface}
\newenvironment{preface}{
    {\noindent \bfseries \Huge \prefacename}
    \begin{center}
        % \phantomsection \addcontentsline{toc}{chapter}{\prefacename} % enable this if you want to put the preface in the table of contents
        \thispagestyle{plain}
    \end{center}%
}


\preface
I wrote these notes principally FOR understanding, they are meant for future reference for a refresher on what I have learnt. \\
Honestly, I don't have much to say, I just thought that a preface would be cool. Let's start our journey now.



\tableofcontents
\pagenumbering{gobble}
\clearpage

\pagenumbering{arabic}
\setcounter{page}{1}

\section{Sentential Logic}
\subsection{Deductive reasoning and logical connectives}
In an \textit{argument}, we arrive at \textit{conclusions} assuming that the \textit{premises} are \textbf{true}. \\ 
Premises are often referred to as conditions, and conclusions are called the outcome. \\
If all the premises are \textbf{true}, then the conclusion should be \textbf{true}. However, for the case where the conclusion is \textbf{false} while the premises are \textbf{true}, the argument is \textbf{invalid}.

\begin{tabular}{| c | c | c |}
\hline
Symbol & Meaning & Description \\
\hline
& & \\
$\lor$ & or & Disjunction \\
& & \\
$\land$ & and & Conjunction \\
& & \\
$\lnot$ & not & Negation \\
\hline
\end{tabular}

\subsection{Truth tables}
\subsection{Variables and sets}
\subsection{Operations on sets}
\subsection{The conditional and biconditional connectives}

\end{document}