\documentclass[../setup.tex]{subfiles}
\graphicspath{{../images}}

\begin{document}

\title{Notes from How to Prove It}
\author{Julian Dominic}
\date{28 July 2022}
\pagenumbering{gobble}
\maketitle
\clearpage

% ===== Define a preface environment =====
\newcommand{\prefacename}{Preface}
\newenvironment{preface}{
    {\noindent \bfseries \Huge \prefacename}
    \begin{center}
        % \phantomsection \addcontentsline{toc}{chapter}{\prefacename} % enable this if you want to put the preface in the table of contents
        \thispagestyle{plain}
    \end{center}%
}


\preface
I wrote these notes principally FOR understanding, they are meant for future reference for a refresher on what I have learnt. \\
Honestly, I don't have much to say, I just thought that a preface would be cool. Let's start our journey now.



\tableofcontents
\pagenumbering{gobble}
\clearpage

\pagenumbering{arabic}
\setcounter{page}{1}



\section{Sentential Logic}
\subsection{Deductive reasoning and logical connectives}
In an \textit{argument}, we arrive at \textbf{valid} \textit{conclusions} assuming that the \textit{premises} are \textbf{true}. \\ 
Premises and conclusiosn are often referred to as conditions and outcomes respectively. \\
If all the premises are \textbf{true}, then the conclusion should be \textbf{true}. However, for the case where the conclusion is \textbf{false} while the premises are \textbf{true}, the argument is \textbf{invalid}.

\subsubsection{Logical Operators}
{
\centering
\begin{tabular}{| c | c | c |}
\hline
Symbol & Meaning & Description \\
\hline
& & \\
$\lor$ & OR & Disjunction \\
& & \\
$\land$ & AND & Conjunction \\
& & \\
$\lnot$ & NOT & Negation \\
\hline
\end{tabular} \\
}



\subsection{Truth tables}
A truth table must be able to represent all possible combinations of the variables, premises and conclusions.\\
\\
{
\centering
\begin{tabular}{|c|c|c|}
\hline
$P$&$Q$&$P\land Q$ \\
\hline
F&F&F \\
F&T&F \\
T&F&F \\
T&T&T \\
\hline
\end{tabular} \\
} 

\paragraph{} In this case, we see that our variables (or statements), $P$ and $Q$ have their individual column to assign a value -- \textbf{True} or \textbf{False} -- to them. \\
We use our logical operators to make a new statement from $P$ and $Q$ which is $P\land Q$, and assign a value to the new statement. \\
\\
It is important note that the number of variables will dictate the number of rows that the truth table will have. Construct a truth table for the following set of variables, $\{P\}$, $\{P, Q\}$, $\{P, Q, R\}$. \\
The pattern that we find is that as the number of variables increases, the number of rows increases two-fold. \\
\[ \text{Number of Rows} = 2^{\text{Number of Variables}} \]
\\
There are some special truth tables where the column for the conclusion always has the same value (either all \textit{true} or all \textit{false}) for every combination of the variables' values. \\
\\
When the conclusion is always \textit{true}, we say that the conclusion's statement is a \textbf{tautologies}. Construct a truth table for $P\lor\lnot P$.\\
Similarly, when the conclusion is always \textit{false}, we say that the conclusion's statement is a \textbf{contradiction}. Construct a truth table for $P \land\lnot P$. \\

\begin{remark}
\textit{Tautologies} and \textit{Contradictions} are not the only laws that govern logic. Do see the logic document for more.
\end{remark}



\subsection{Variables and sets}
\subsubsection{Sets}
A set is a collection of elements. The order of the elements in the set does not matter. If an element appears more than once, it is still the same set. \\
\[ \{3, 7, 14\} \equiv \{7, 3, 14\} \equiv \{14, 3, 7, 7\} \]
\\
When the set is infinite or has too many elemnts to list, we will define it explicitly. Suppose we have the following set $P$. \\

\[ P = \{x \ |\  x \text{ is a prime number}\} \]
\\
How we read $P = \{x | x \text{is a prime number}\}$ is ``$P$ is equal to the set of all $x$ such that $x$ is a prime number.'' Which also means that $P$ contains all values of $x$ that make the statement ``$x$ is a prime number'' true. \\
Some direct translations of the symbols into words would be (i) ``\{\}'' means ``the set of'', and (ii) ``$|$'' means ``such that''. \\
\\
Sets like $P$ have an \textbf{elementhood test} for the set; in this case, the \textit{elementhood test} is being a prime number. Any value of $x$ that makes the statement come out true, passes the test and is an element of the set. \\
The \textbf{Truth set} of a statement $P(x)$ is the set of all values of $x$ that make the statement $P(x)$ true. In other words, it is the set defined by using the statement $P(x)$ as the elementhood test: \\
\[ \text{Truth set of } P(x) = \{x \ | \ P(x)\} \]

\subsubsection{Understanding variables used in sets}
There are two types of variables that can appear in a set; \textbf{Free variables} and \textbf{Bound variables}. \textit{Free variables} are variables that will make the statement either \textit{True} or \textit{False} while \textit{Bound variables} are variables whose values we do not need to know (they can be considered \textit{dummy variables}). Lets consider the following example, \\
\\
 \[ y \in \{x \ | \ x^2 < 9\} \] 
\\
For any number $y$, to verify $y \in \{x \ | \ x^2 < 9\}$, we have to check $y^2 < 9$. Since $y \in \{x \ | \ x^2 < 9\}$ is just a roundabout way of saying $y^2 < 9$, it follows that we do not need to care about the value of $x$. Rather, only the value of $y$ is required. Thus, we can say that $y$ is a \textit{free variable} while $x$ is a \textit{bound variable}. As such, we can go further and replace $x$ with any other variable except $y$ because we do not need to care what $x$ is. As such, it can even be $y \in \{w\ | \ w^2 < 9\}$ where $y$ is \textit{free} and $w$ is \textit{bound}. Notice that $x^2 < 9$ makes $x$ a \text{free variable}. We can say that that statement $P(x)$ in $\{x \ | \ P(x)\}$ \textbf{binds} the variable $x$. \\


\begin{remark}
In general, the statement $y \in \{x\ |\ P(x)\} \Rightarrow P(y)$ and $y \notin \{x\ |\ P(x)\} \Rightarrow \lnot P(y)$. \\
It is also important to note that ${x\ |\ P(x)}$ is not a statement. It is a set because of the curly brackets ``\{\}''
\end{remark}
\clearpage

The \textbf{Universe of Discourse}, $U$, is the set of all possible values for the variables. We can say things such as $\{x \in U\ |\ P(x)\}$: The set of all $x$ in $U$ such that $P(x)$. For a set that has the \textit{universe of discourse} defined, an element of the set has to pass two tests, $x \in U$ and $P(x)$. Therefore, in general, $y \in \{x\in A\ |\ P(x)\} \Rightarrow y\in A \land P(y)$. \\
If $P(x)$ is false for every possible value of $x$, it yields a truth set with no elements. As such, we get the \textbf{empty set}/\textbf{null set}; $\emptyset$ or $\{\}$ where the contents inside the curly brackets are blank. For example, \\
\[\{x\in\mathbb{Z} \ |\ x \neq x\} = \emptyset = \{\}\]

\begin{remark}
$\emptyset \neq \{\emptyset\}$. $\emptyset$ is a set while $\{\emptyset\}$ is a set of a set.
\end{remark}
\subsection{Operations on sets}
\subsection{The conditional and biconditional connectives}

\end{document}